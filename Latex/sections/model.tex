The simplest model of an isolated population has linear birth and death terms (that is, the per capita birth and death rates are constant).
This model is a classic but gives the outcome of population explosion with some probability\cite{Nisbet1982}. %it’s not a time, it’s a fraction of the instances
%, as probably is the case with constant birth/death (immigration/emigration) or any combination of these two. [find examples]
To mathematically curb this infinite growth, and to biologically allow for intraspecies interactions, a non-linear term is required, and a quadratic is the easiest non-linearity to handle.
The difference between per capita birth and death gives some rate constant $r$, and this rate constant is inhibited by the density of the population, hence a decrease by $n/K$, giving the desired quadratic term.
This is exactly the logistic equation \ref{logistic}.

Extinction occurs at $n=0$, an unstable fixed point of the logistic equation, whereas there is a stable fixed point at $n=K$.
%The logistic equation \ref{logistic} has fixed points at extinction and the carrying capacity, $n=0$ and $n=K$ respectively.
Common practice in dynamical systems analysis is to rescale variables to remove parameters and simplify the system.
Since we are dealing with continuous time we can remove the rate constant from our equation; we do so by rescaling the time by $r$.
Similarly, in the deterministic equation we could rescale $n$ by $K$ and have no remaining parameters.
However, in the stochastic version we cannot apply this latter rescaling, because of the implicit population scale of $\pm1$ organism for each birth/death event. %or “because the integer number of organisms has an implicit population scale of 1.” %organism/individual

Here we have assumed that the stochasticity comes from the discretization of the population, that it must exist at integer values, in opposition with the results of a deterministic model like equation \ref{logistic}.
Such stochasticity is termed demographic noise.
Instead of a birth rate $b_n$ we assume that each birth event is independent and distributed exponentially with a probability $b_n\,dt$ of occuring in each infinitesimal time interval $dt$, and this is similarly assumed for death events. %too technical? Or esoteric?
In this paper we use the birth rate
\begin{equation}
 b_n = \Big(1 + \frac{\delta}{2}\Big)\,r\,n - \frac{q\,r}{K}n^2% = r\,n\left(1+\frac{\delta}{2}-q\,n/K\right)
\label{birth}
\end{equation}
and death rate
\begin{equation}
 d_n = \frac{\delta}{2}\,r\,n + \frac{(1-q)r}{K} n^2 = r\,n\left(\frac{\delta}{2}+(1-q)\,n/K\right).
\label{death}
\end{equation}
Note that we introduce two new parameters in our equations \ref{birth} and \ref{death}: $q\in[0,1]$ shifts the nonlinearity between the death term and the birth term whereas the parameter $\delta\in[0,\infty)$ establishes a scale for the contribution of linear terms in both the birth and death rates.
We include the parameter $\delta$ to account for the stochastic relevance of the absolute values of the per capita birth and death rates, but in the deterministic limit only their difference $r$ affects the dynamics of the system.
Parameter $q$ describes where the intraspecies inhibition acts: a high $q$ near unity implies competition for resources and a decreased effective birth rate, whereas a low $q$ near zero reflects more direct conflict, with intraspecies interactions resulting in greater death rates of organisms.
%In this formulation we can vary the strength of the density-dependence in the per capita death and birth rates by the factor $q$.
It can be readily checked that $b_n-d_n$ recovers the right-hand side of equation \ref{logistic} where, as per design, the new parameters $q$ and $\delta$ do not appear.
The choice of these parameters specifies a particular model and has consequences on the MTE.
%We will find that this choice has consequences on the MTE. %and pdf

The model described above has one other notable feature.
Except at $q=0$, there is a population at which the competition brings the effective birth rate to zero.
This is the maximum size the population can achieve, and we define this cutoff as
%This limits the population to a maximal size $N = \lceil n_{max}\rceil$, where $n_{max}$ is defined as the population size such that $b_{n_{max}}=0$.
%From equation \ref{birth} we find that
\begin{equation}
N = \frac{K}{q}\Big(\,1 + \frac{\delta}{2}\,\Big).
\label{maxN}
\end{equation}
Therefore we limit our calculations to the biologically relevant range $n\in[0,N]$. % and, for completeness to our study, we can readily check that for our range of parameters $N\geq K$.
Already it is evident that the ``hidden" parameters of $\delta$ and $q$ have an effect on the system, as different values of the parameters will naturally define a range of states accessible in the model.
Note that so long as $q\leq 1$ the death rate is positive semi-definite for the domain of interest. % as defined previously does not imply any necessary subtle manipulation of the population range since the death rate is always positive (except at $n=0$) in the range of $q$ and $\delta$ described earlier.
%The lower bound of the population range for all models is at our unstable fixed point representing an extinct species $n=0$.
At $n=0$ both the birth and death rates go to zero: this is the stochastic absorbing state.
