A probability distribution function is a useful mathematical tool to describe the state of a dynamical system.
We define $P_n(t)$ as the probability that the population is composed of $n$ organisms at time $t$.
The evolution of the distribution in our single birth and death process is captured in the master equation
\begin{equation}
\frac{dP_n}{dt} =  b_n P_{n-1}(t) + d_n P_{n+1}(t) - (b_n+d_n)P_n(t).
\label{master-eqn}
\end{equation}
Note that ultimately at large times the probability of being at population size $n\neq 0$ decays to zero, as more and more of the probability gets drawn to the absorbing state.
%This is due to the stochasticity of the births and deaths and the nature of the absorbing state $n=0$ with no possibility of recovery [reference probability distribution leaking to zero definitely].
%Although this is an important property of this model, it is difficult to describe any dynamics of our model with such a distribution.
Prior to reaching extinction, the system tends toward a quasi-stationary distribution.

We are interested in this conditional probability distribution function $P_n^c$: the probability distribution of the population conditional to not being in the extinct state: %stationary
\begin{equation}
P_n^c = \frac{P_n}{1-P_0}.
\end{equation}
This dynamics of the distribution are described in a slightly different master equation than equation \ref{master-eqn}:
\begin{equation}
\frac{dP_n^c}{dt} =  b_{n-1}P_{n-1}^c(t) + d_{n+1}P_{n+1}^c(t) - (b_n + d_n - P_1^c(t)d_1)P_n^c(t).
\label{masters2}
\end{equation}
After an initial transient period, this conditional probability will stabilize to some steady $\tilde{P}^c_n$ for which $d\tilde{P}_n^c/dt=0$.
The steady state of this distribution is referred to as the quasi-stationary distribution, not to be confused with the true stationary distribution of the population which is the state where $\tilde{P}_n(t \rightarrow \infty)=\delta_{n,0}$.

\begin{figure}[ht!]
  \centering
  \subfloat[\emph{Mean of the probablity distribution of the population size, with capacity $K=100$.}]{\includegraphics[width=0.5\textwidth]{Figure1-A}\label{qsd:q}}
  \hfill
  \subfloat[\emph{Variance of the probability distribution of the population.}]{\includegraphics[width=0.5\textwidth]{Figure1-B}\label{qsd:delta}}
  \caption{\emph{Probability distribution of the population} The conditional probability distribution functions as found using the quasi-stationary distribution algorithm. [NEED TO ADD SOMETHING ABOUT THE PROBABILITY DISTRIBUTION BEING NOT PERFECTLY GAUSSIAN]}
  \label{qsd}
%The range along the horizontal axis does not fully cover the population, it is truncated to show the relevant region of the distribution. In fact each curve has a different range as the parameters $\delta$ and $q$ vary the maximum population size according to equation \ref{maxN}.
\end{figure}

One way to obtain the quasi-stationary distribution is to exploit equation \ref{masters2} in an algorithm in which we iteratively calculate the change in the distribution $\Delta P^c_n$ in an arbitrarily small time interval $\Delta t$ until all change in the distribution is negligible \cite{Nisbet1982}.
%We start with an arbitrary initial distribution $P^c_n(0)$ and calculate the change $\Delta P^c_n$ for each $n$ in an arbitrarily small time interval $\Delta t$.
%Thus we obtain a new distribution $P^c_n(\Delta t)$.
%We continue this iterative procedure until the changes in the distribution $|\Delta P^c_n|$ are below a certain threshold.
%Ideally, this iterative process would continue until all $\Delta P^c_n=0$.
%The accuracy of the algorithm is determined by the time interval $\Delta t$ and reducing this value increases the runtime of the algorithm as more steps are needed to get a steady state solution.
Decreasing the time step $\Delta t$ increases both the accuracy and the runtime, such that an arbitrarily accurate distribution takes a prohibitively long time to calculate.
We settle for $\Delta P^c_n<\epsilon = 10^{-16}$. %CHECK THIS%Maybe have this in the figure caption - figure goes to probabilities of 10^-20, which is fine, as this -16 is the maximal difference, not necessarily the resolution at the extremes


Results of this algorithm, for different values of $q$ and $\delta$, are presented in Figure \ref{qsd}.
Increasing the value of $\delta$ shifts the mode toward the anterior of the distribution and spreads the distribution out, increasing the variance.
Decreasing $q$ has a similar effect. %decreasing q gives broader and anterior
%The maximal value of a distribution shifts as we vary either of these parameters.
