\begin{abstract}
A quadratic rate equation is the simplest way to incorporate nonlinearity, and therefore interactions, in models of population dynamics. This leads to the popular logistic equation, also known as the Verhulst equation.
In this paper, we consider different birth and death rates that lead to the same deterministic dynamics.
Using the stochastic results of mean time to extinction and quasi-steady state distribution, we show that ``hidden’’ parameters that do not show up in the deterministic dynamics nevertheless have a profound contribution on these stochastic metrics.
We conclude that researchers must take care when formulating their problems, and that the underlying stochastic structure should be a conscious choice motivated by the biology/physics rather than one of convenience.
%In so doing,
In the process of our investigation, we also evaluate a number of common approximations employed in stochastic analysis, and find that in most cases the WKB and Fokker-Planck approximations recover the leading order scaling of the extinction time but fail at the higher order.
\end{abstract}
