\documentclass[a4paper,10pt]{article}
%\usepackage[utf8x]{inputenc}
%\usepackage{wrapfig}
\usepackage{graphicx}    % needed for including graphics e.g. EPS, PS
%\graphicspath{{C:/Users/zilmangroup/Documents/mathematica/images/}} %%%%%%%%%%%%%%%
\graphicspath{{/home/jrothschild/Research/PopDyn_variation/Figures/}} %%%%%%%%%%%%%%%
\usepackage{amsmath, amsthm, amssymb, braket, color} %%%%%%%%%%%%%%%%
\usepackage[usenames,dvipsnames]{xcolor} %%%%%%%%%%%%%
\usepackage{subfig} %For subfigures
\usepackage[normalem]{ulem} %for striking out text
\usepackage{subfiles} %subfile packages
\graphicspath{{images/}{../images/}} %where the images are
\usepackage[numbers,sort&compress]{natbib} %compressed bibliography style

\numberwithin{equation}{section} %%%%%%%%%%%%%%%
\topmargin -1.5cm        % read Lamport p.163
\oddsidemargin -0.04cm   % read Lamport p.163
\evensidemargin -0.04cm  % same as oddsidemargin but for left-hand pages
\textwidth 16.59cm
\textheight 21.94cm
%\pagestyle{empty}       % Uncomment if don't want page numbers
\parskip 0pt           % sets spacing between paragraphs
%\renewcommand{\baselinestretch}{1.5}     % Uncomment for 1.5 spacing between lines
\parindent 8pt          % sets leading space for paragraphs

%opening
\title{Moving the nonlinearity between birth and death.}
\author{MattheW Badali, Jeremy Rothschild, Anton Zilman}

\begin{document}

\maketitle

%\begin{abstract}
Skip it for now
%\end{abstract}

\section{Introduction}

\subfile{sections/introduction.tex}

\section{Model}

\subfile{sections/model.tex}

\section{Quasi-stationary Probability Distribution Function}

\subfile{sections/quasi-stationary_pdf.tex}

\section{Exact Mean Time to Extinction}% - 

\subfile{sections/mte.tex}

\section{Approximations}

\subfile{sections/approximations.tex}

\section{Discussion}
%QUESTION: what is a “model”, in our language? A set of parameters?
%ANSWER: we have one framework that encapsulates many models

\subfile{sections/discussion.tex}

\bibliographystyle{unsrt}
\bibliography{library.bib,paper1-6final.bib}
\end{document}


